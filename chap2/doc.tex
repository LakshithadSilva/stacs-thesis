%
% Thesis - Chapter 2
% Lakshitha de Silva
% School of Computer Science
% University of St Andrews, Scotland, UK
% 2014
%


\chapter{Context Survey}
\label{chap:background}
Numerous approaches have been proposed over the years either to prevent architecture erosion or to detect and restore eroded architectures. This chapter presents a survey of those approaches, which include techniques, tools and processes. They are classified primarily into three generic three categories that attempt to minimise, prevent and repair architecture erosion. Within these broad categories, each approach is further broken down  to reflect the high-level strategies adopted to tackle erosion. Some of these strategies in turn contain sub-categories under which survey results are presented. Merits and weaknesses of each strategy is discussed, with the argument that no single strategy can address the problem of erosion. The chapter concludes by presenting a case for further work in developing a holistic and practical approach for controlling architecture erosion.


\section{Introduction}
\subsection{Terminology}
\subsection{Other surveys}

\section{Classification}

\section{Discussion}

\section{Conclusions}
